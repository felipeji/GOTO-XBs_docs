%% Generated by Sphinx.
\def\sphinxdocclass{report}
\documentclass[letterpaper,10pt,english]{sphinxmanual}
\ifdefined\pdfpxdimen
   \let\sphinxpxdimen\pdfpxdimen\else\newdimen\sphinxpxdimen
\fi \sphinxpxdimen=.75bp\relax
\ifdefined\pdfimageresolution
    \pdfimageresolution= \numexpr \dimexpr1in\relax/\sphinxpxdimen\relax
\fi
%% let collapsible pdf bookmarks panel have high depth per default
\PassOptionsToPackage{bookmarksdepth=5}{hyperref}

\PassOptionsToPackage{booktabs}{sphinx}
\PassOptionsToPackage{colorrows}{sphinx}

\PassOptionsToPackage{warn}{textcomp}
\usepackage[utf8]{inputenc}
\ifdefined\DeclareUnicodeCharacter
% support both utf8 and utf8x syntaxes
  \ifdefined\DeclareUnicodeCharacterAsOptional
    \def\sphinxDUC#1{\DeclareUnicodeCharacter{"#1}}
  \else
    \let\sphinxDUC\DeclareUnicodeCharacter
  \fi
  \sphinxDUC{00A0}{\nobreakspace}
  \sphinxDUC{2500}{\sphinxunichar{2500}}
  \sphinxDUC{2502}{\sphinxunichar{2502}}
  \sphinxDUC{2514}{\sphinxunichar{2514}}
  \sphinxDUC{251C}{\sphinxunichar{251C}}
  \sphinxDUC{2572}{\textbackslash}
\fi
\usepackage{cmap}
\usepackage[T1]{fontenc}
\usepackage{amsmath,amssymb,amstext}
\usepackage{babel}



\usepackage{tgtermes}
\usepackage{tgheros}
\renewcommand{\ttdefault}{txtt}



\usepackage[Bjarne]{fncychap}
\usepackage{sphinx}

\fvset{fontsize=auto}
\usepackage{geometry}


% Include hyperref last.
\usepackage{hyperref}
% Fix anchor placement for figures with captions.
\usepackage{hypcap}% it must be loaded after hyperref.
% Set up styles of URL: it should be placed after hyperref.
\urlstyle{same}


\usepackage{sphinxmessages}
\setcounter{tocdepth}{1}



\title{GOTO\sphinxhyphen{}XBs}
\date{Sep 23, 2023}
\release{}
\author{Felipe Jiménez\sphinxhyphen{}Ibarra}
\newcommand{\sphinxlogo}{\vbox{}}
\renewcommand{\releasename}{}
\makeindex
\begin{document}

\ifdefined\shorthandoff
  \ifnum\catcode`\=\string=\active\shorthandoff{=}\fi
  \ifnum\catcode`\"=\active\shorthandoff{"}\fi
\fi

\pagestyle{empty}
\sphinxmaketitle
\pagestyle{plain}
\sphinxtableofcontents
\pagestyle{normal}
\phantomsection\label{\detokenize{index::doc}}


\sphinxAtStartPar
GOTO\sphinxhyphen{}XBs is a web platform for follow\sphinxhyphen{}up of X\sphinxhyphen{}ray binaries (XBs) within GOTO data. It provides access to light curves for all known X\sphinxhyphen{}ray binaries, constructed from all the observed epochs and across the 4 GOTO bands. The object database can be updated with newly discovered X\sphinxhyphen{}ray binaries, triggering the automatic generation of historical light curves for these objects.

\sphinxAtStartPar
The platform also features a range of analysis tools for working with light curves in real\sphinxhyphen{}time. These tools include data filtering, phase folding,  smoothing within user\sphinxhyphen{}defined time windows, periodogram analysis, plot generation, and more.

\sphinxAtStartPar
Users can also download the data, both raw or previously filtered or smoothed light curves.

\sphinxAtStartPar
The platform is built on a Django framework. So far it consists of two applications:
\begin{itemize}
\item {} 
\sphinxAtStartPar
\sphinxstylestrong{dashboard}: a list of the sources being followed. They are organized in a table that makes it easy to find them.

\item {} 
\sphinxAtStartPar
\sphinxstylestrong{source\_viewer}:  produces an individual webpage for each object with its information. It use \sphinxcode{\sphinxupquote{plotly}} for creating an interactive plot of the object light curve in GOTO and it allows to download the data.

\end{itemize}


\chapter{Contents}
\label{\detokenize{index:contents}}
\sphinxstepscope


\section{Requirements}
\label{\detokenize{requirements:requirements}}\label{\detokenize{requirements::doc}}
\sphinxAtStartPar
GOTO\sphinxhyphen{}XBs is built on the Django framework, and the requirements necessary to use it are:
\begin{itemize}
\item {} 
\sphinxAtStartPar
\sphinxstylestrong{python 3.8.x}

\item {} 
\sphinxAtStartPar
\sphinxstylestrong{conda} or \sphinxstylestrong{virtualenv + virtualenvwrapper}

\end{itemize}

\sphinxAtStartPar
Check  or  in case you need it. More details on the dependencies needed and how to install them in the Quick Start Guide ({\hyperref[\detokenize{quick_start:virtual-env}]{\sphinxcrossref{\DUrole{std,std-ref,std,std-ref}{GOTO\sphinxhyphen{}XBs Environment}}}}).

\sphinxAtStartPar
\sphinxstylestrong{Ready!?} Continue to the {\hyperref[\detokenize{quick_start::doc}]{\sphinxcrossref{\DUrole{doc}{Quick Start Guide}}}}!

\sphinxstepscope


\section{Quick Start Guide}
\label{\detokenize{quick_start:quick-start-guide}}\label{\detokenize{quick_start::doc}}

\subsection{Downloading GOTO\sphinxhyphen{}XBs Repository}
\label{\detokenize{quick_start:downloading-goto-xbs-repository}}
\sphinxAtStartPar
The source code for GOTO\sphinxhyphen{}XBS is hosted on the GOTO GitHub page. To developing GOTO\sphinxhyphen{}XBs locally, you can clone the repository in : \sphinxurl{https://github.com/GOTO-OBS/goto-xbs.git}


\subsection{GOTO\sphinxhyphen{}XBs Environment}
\label{\detokenize{quick_start:goto-xbs-environment}}\label{\detokenize{quick_start:virtual-env}}
\sphinxAtStartPar
GOTO\sphinxhyphen{}XBs requires a specific environment with the necessary dependencies to run smoothly. We maintain two options for creating this environment: using Conda or virtualenv


\subsubsection{Using Conda}
\label{\detokenize{quick_start:using-conda}}\begin{enumerate}
\sphinxsetlistlabels{\arabic}{enumi}{enumii}{}{.}%
\item {} 
\sphinxAtStartPar
\sphinxstylestrong{Install Conda}:

\sphinxAtStartPar
If you haven’t already, download and install Conda. We suggest to use Miniconda, a minimal installer for Conda, from the \sphinxhref{https://docs.conda.io/en/latest/miniconda.html}{official Miniconda website}.

\item {} 
\sphinxAtStartPar
\sphinxstylestrong{Create a Conda Environment}:

\sphinxAtStartPar
Open your terminal and create a new Conda environment named ‘goto\sphinxhyphen{}xbs’ using \sphinxcode{\sphinxupquote{goto\sphinxhyphen{}xbs\sphinxhyphen{}env.yml}} file (provided in the root directory):

\begin{sphinxVerbatim}[commandchars=\\\{\}]
conda\PYG{+w}{ }env\PYG{+w}{ }create\PYG{+w}{ }\PYGZhy{}f\PYG{+w}{ }goto\PYGZhy{}xbs\PYGZhy{}env.yml
\end{sphinxVerbatim}

\item {} 
\sphinxAtStartPar
\sphinxstylestrong{Activate the Environment}:

\sphinxAtStartPar
To activate the newly created environment:

\begin{sphinxVerbatim}[commandchars=\\\{\}]
conda\PYG{+w}{ }activate\PYG{+w}{ }goto\PYGZhy{}xbs
\end{sphinxVerbatim}

\end{enumerate}

\sphinxAtStartPar
and you should now be in the ‘goto\sphinxhyphen{}xbs’ environment.


\subsubsection{Using virtualenv}
\label{\detokenize{quick_start:using-virtualenv}}\begin{enumerate}
\sphinxsetlistlabels{\arabic}{enumi}{enumii}{}{.}%
\item {} 
\sphinxAtStartPar
\sphinxstylestrong{Install Python}:

\sphinxAtStartPar
You will need Python 3.8.x installed on your system (check \sphinxhref{https://www.python.org/downloads/}{official Python website}. in case you need it)

\item {} 
\sphinxAtStartPar
\sphinxstylestrong{Create a virtualenv}:

\sphinxAtStartPar
Open your terminal and navigate to the root directory of your GOTO\sphinxhyphen{}XBs project. Create a virtual environment named ‘goto\sphinxhyphen{}xbs’ using the following command:

\begin{sphinxVerbatim}[commandchars=\\\{\}]
python\PYG{+w}{ }\PYGZhy{}m\PYG{+w}{ }venv\PYG{+w}{ }goto\PYGZhy{}xbs
\end{sphinxVerbatim}

\item {} 
\sphinxAtStartPar
\sphinxstylestrong{Activate the Environment}:
\begin{quote}

\begin{sphinxVerbatim}[commandchars=\\\{\}]
\PYG{n+nb}{source}\PYG{+w}{ }goto\PYGZhy{}xbs/bin/activate
\end{sphinxVerbatim}
\end{quote}

\sphinxAtStartPar
You should now be in the ‘goto\sphinxhyphen{}xbs’ virtual environment.

\end{enumerate}
\begin{enumerate}
\sphinxsetlistlabels{\arabic}{enumi}{enumii}{}{.}%
\setcounter{enumi}{2}
\item {} 
\sphinxAtStartPar
\sphinxstylestrong{Installing Dependencies}:

\end{enumerate}

\sphinxAtStartPar
You can now install the project’s dependencies using the \sphinxcode{\sphinxupquote{requirements.txt}} file:

\begin{sphinxVerbatim}[commandchars=\\\{\}]
pip\PYG{+w}{ }install\PYG{+w}{ }\PYGZhy{}r\PYG{+w}{ }requirements.txt
\end{sphinxVerbatim}

\sphinxAtStartPar
Your environment is now set up with all the necessary dependencies for GOTO\sphinxhyphen{}XBs. You can proceed to run and use the platform.


\subsection{Settings Files}
\label{\detokenize{quick_start:settings-files}}
\sphinxAtStartPar
Settings are organized into three separate files: \sphinxcode{\sphinxupquote{dev\_settings.py}}, \sphinxcode{\sphinxupquote{production\_settings.py}}, and \sphinxcode{\sphinxupquote{base\_settings.py}}.
\begin{itemize}
\item {} 
\sphinxAtStartPar
\sphinxcode{\sphinxupquote{dev\_settings.py}}: is the app settings used when running in a development environment.

\item {} 
\sphinxAtStartPar
\sphinxcode{\sphinxupquote{production\_settings.py}}: is employed when deploying the system in a production environment.

\item {} 
\sphinxAtStartPar
\sphinxcode{\sphinxupquote{base\_settings.py}}: contains settings common to both the development and production environments. It is loaded at the start of both \sphinxtitleref{dev\_settings.py} and \sphinxtitleref{production\_settings.py}, reducing redundancy and ensuring consistency.

\end{itemize}

\sphinxAtStartPar
To switch between development and production settings, you need to define the \sphinxcode{\sphinxupquote{DJANGO\_SETTINGS\_MODULE}} system variable.

\sphinxAtStartPar
To activate development settings, use the following command in your terminal:

\begin{sphinxVerbatim}[commandchars=\\\{\}]
\PYGZdl{} export DJANGO\PYGZus{}SETTINGS\PYGZus{}MODULE=goto\PYGZus{}xbs.dev\PYGZus{}settings
\end{sphinxVerbatim}

\sphinxAtStartPar
To activate production settings, use this command:

\begin{sphinxVerbatim}[commandchars=\\\{\}]
\PYGZdl{} export DJANGO\PYGZus{}SETTINGS\PYGZus{}MODULE=goto\PYGZus{}xbs.production\PYGZus{}settings
\end{sphinxVerbatim}

\sphinxAtStartPar
You can verify the current state of the \sphinxcode{\sphinxupquote{DJANGO\_SETTINGS\_MODULE}} variable echoing in the terminal by:

\begin{sphinxVerbatim}[commandchars=\\\{\}]
\PYGZdl{} echo \PYGZdl{}DJANGO\PYGZus{}SETTINGS\PYGZus{}MODULE
\end{sphinxVerbatim}

\sphinxAtStartPar
This will display the currently active settings module.

\sphinxstepscope


\section{Databases Description}
\label{\detokenize{databases:databases-description}}\label{\detokenize{databases::doc}}
\sphinxAtStartPar
GOTO\sphinxhyphen{}XBs utilizes an SQLite database with two primary tables:


\subsection{Source Table}
\label{\detokenize{databases:source-table}}
\sphinxAtStartPar
The \sphinxtitleref{Source} table contains fundamental information about the X\sphinxhyphen{}ray binaries (XBs) being tracked. It is structured with the following columns:
\begin{itemize}
\item {} 
\sphinxAtStartPar
\sphinxcode{\sphinxupquote{source\_id}}: a unique identifier for each source.

\item {} 
\sphinxAtStartPar
\sphinxcode{\sphinxupquote{name}}: the name of the source.

\item {} 
\sphinxAtStartPar
\sphinxcode{\sphinxupquote{pretty\_name}}: a human\sphinxhyphen{}readable name for the source.

\item {} 
\sphinxAtStartPar
\sphinxcode{\sphinxupquote{ra}}: Right Ascension (RA) coordinate of the source.

\item {} 
\sphinxAtStartPar
\sphinxcode{\sphinxupquote{dec}}: Declination (Dec) coordinate of the source.

\end{itemize}


\subsection{Photometry Table}
\label{\detokenize{databases:photometry-table}}
\sphinxAtStartPar
The \sphinxtitleref{Photometry} table stores photometric data obtained from the GOTO database. It includes the following columns:
\begin{itemize}
\item {} 
\sphinxAtStartPar
\sphinxcode{\sphinxupquote{source}}: a unique identifier linking the data to a specific source. This field serves as a foreign key referencing the \sphinxtitleref{source\_id} in the \sphinxtitleref{Source} table.

\item {} 
\sphinxAtStartPar
\sphinxcode{\sphinxupquote{date}}: date of the observation.

\item {} 
\sphinxAtStartPar
\sphinxcode{\sphinxupquote{mjd}}: Modified Julian Date (MJD) of the observation.

\item {} 
\sphinxAtStartPar
\sphinxcode{\sphinxupquote{mag}}: magnitude.

\item {} 
\sphinxAtStartPar
\sphinxcode{\sphinxupquote{mag\_err}}: error associated with the magnitude.

\end{itemize}


\chapter{Indices and tables}
\label{\detokenize{index:indices-and-tables}}\begin{itemize}
\item {} 
\sphinxAtStartPar
\DUrole{xref,std,std-ref}{genindex}

\item {} 
\sphinxAtStartPar
\DUrole{xref,std,std-ref}{modindex}

\item {} 
\sphinxAtStartPar
\DUrole{xref,std,std-ref}{search}

\end{itemize}



\renewcommand{\indexname}{Index}
\printindex
\end{document}